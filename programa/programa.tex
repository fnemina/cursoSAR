\documentclass[a4paper,12pt]{article}
\input{../preamble.tex}
\usepackage{float}
\usepackage{enumitem}
%\includeonly{guia1}
\graphicspath{{./figs/}}
% Renombro lo capitulos
\addto\captionsspanish{\renewcommand{\chaptername}{Clase}}


\title{ {\large Nivel 2} \\ Introducción al color del Oceano}
\author{}

%\author[*]{Laura Rouco}


%\titlepic{\vfill\includegraphics[height=1.5cm]{../figs/logo-ministerio.png}\hfill\includegraphics[height=1.5cm]{../figs/logo-conae.png}}
\pagenumbering{gobble}
\date{}

\begin{document}
%\frontmatter
\maketitle
%\titlepage
%\tableofcontents
%\include{prefacio}
\vskip-2.2cm
%\mainmatter
El Curso Nivel 2: Introducción al color del oceano en su Modalidad Virtual está destinado a usuarios de imágenes satelitales que hayan realizado y aprobado el Curso Nivel 1: Introducción a Teledetección o tener conocimientos equivalentes al mencionado curso en teledetección. El objetivo de esta capacitación es que los participantes adquieran los principales conceptos y aplicaciones de las imágenes de color del oceano. 
\section*{Organización general}
\begin{itemize}
    \item Modalidad: a distancia, a través de la plataforma virtual de la UEFM.
    \item Duración: 5 clases.
    \item Carga horaria: 25 horas.
    \item Requisitos: Haber realizado y aprobado el Curso Nivel I: Introducción a Teledetección o tener conocimientos equivalentes al mencionado curso en teledetección. Contar con una PC con Windows 7 o superior, Linux. Procesador i3 o superior, 8GB de memoria RAM, 30GB de espacio en disco disponible y una conexión a internet de 1 Mbit/s
        
\end{itemize}

\section*{Contenidos}
\begin{enumerate}[leftmargin=*, label={Clase \arabic* -}]
    \item \emph{Introducción:}
    Sensores remotos en color del oceano. Satélites y órbitas. Satélites geosincronicos. Satélites heliosincronicos. Técnicas de adquisición de datos. Niveles de procesamiento. Archivos de datos de color del oceano.
    \item \emph{Radiación electromagnética:}
    Introducción. Ondas electromagneticas. Espectro electromagnético. Descripción de radiación. Magnitudes físicas.
    \item \emph{Propiedades de la atmósfera:}
    Descripción de la atmósfera. Moleculas: emisión y absorción. Scattering. Atenuación en la atmósfera. Ecuación de transferencia radiativa. Soluciones a la ecuación de transferencia radiativa. Transmitacia difusa.
    \item \emph{Color del océano:}
    Absorción y scattering del phitoplackton, particulas y material disuelto. Instrumentos usados en color del oceano. Correcciones atmosféricas. Calibración y validación. Modelos empíricos y semiempíricos. 
    \item \emph{Misión SABIA-Mar y aplicaciones:}
    Plan Espacial Nacional. Misión SABIA-Mar. Misiones de color del oceano. Aplicaciones de los datos de color del oceano.
\end{enumerate}


\end{document}
