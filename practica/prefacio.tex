
\chapter{Descripción del curso}

El curso busca brindar conceptos básicos de interpretación visual y procesamiento de imágenes de radar de apertura sintética. Para ello se trabajara en una modalidad teórico-práctica que incluye una introducción teórica al tema, un ejemplo práctico sobre este y una evaluación corta sobre los conceptos aprendidos.

\section*{Carga horaria}
El curso tiene una carga horaria de 25 horas reloj repartido en 5 clases semanales.

\section*{Requisitos} Haber cursado el curso SoPI I o acreditar conocimientos
equivalentes de teledetección.

\section*{Aprobación} El curso brindará certificado de \emph{aprobación}. Para obtenerlo el alumno deberá aprobar todos los cuestionarios de evaluación y completar una encuesta de inicio y una encuenta final sobre el curso.

\section*{Bibliografía}
\textbf {\large Texto:} \emph{Remote Sensing with Imaging Radar},
5\textsuperscript{th} Edition
\textbf {Autor:} John A. Richards; \\

\section*{Objetivos del curso}
Al finalizar el curso, el alumno podrá:
\begin{enumerate} \itemsep-0.4em
  \item Interpretar imágenes SAR en las bandas X, L y C.
  \item Obtener y procesar imágenes SAR de distintas fuentes.
  \item Interpretar imágenes SAR full polarimétricas en la descomposición de Pauli.
  \item Conocer distinas áreas de aplicación de las imágenes SAR.
\end{enumerate}

\section*{Programa}

\subsection*{Clase 1: introducción al radar}
Espectro electromagnético. Operación de un radar. Formación de imágenes SAR. Geometría de adquisición. Modos de adquisición. Comparación con imágenes ópticas.

\subsection*{Clase 2: interacciones con el blanco}
Interpretación visual de imágenes SAR. Textura y geometría de coberturas típicas. Coeficiente de retrodispersión. Bandas L, C y X. Formás de interacción con el blanco: doble rebote, en volumen y especular. Resolución espacial.

\subsection*{Clase 3: Procesamiento y speckle}
Calibración de imágenes SAR. Ruido speckle y multilooking. Distorciones geométricas. Proyección en rango y en el terreno. Niveles de procesamiento de imágenes SAR.

\subsection*{Clase 4: polarimetría}
Ondas electromagnéticas y polarización. Imágenes full polarimetricas. Matriz de scatter y backscatter. Descomposiciónes polarimétricas. Descomposición de Pauli.

\subsection*{Clase 5: misión SAOCOM y aplicaciones}
Misiones SAR. Datos históricos y actuales. Aplicaciones de los datos SAR. Plan espacial nacional. Misión SAOCOM.

\section*{Cronograma}
% Fecha de inicio del curso
\SetDate[03/04/2018]
\begin{longtable}[h!]{ c  c  }
%\normalsize % The size of the table text can be changed depending on content. Remove if desired.
%\begin{tabular}{ | c | c | }
\toprule
\textbf{Día} & \textbf{Contenido} \\
\midrule
\mydate\today & \begin{minipage}{.80\textwidth}
\begin{itemize}
    \vspace{1mm}
	\item Actividad: Encuesta de inicio del curso. Descara de materiales y software.
    \vspace{1mm}
\end{itemize}
\end{minipage} \\
\midrule
\AdvanceDate[7] \mydate\today & \begin{minipage}{.80\textwidth}
\begin{itemize}
    \vspace{1mm}
  \item Video teórico: introducción al radar.
  \item Práctica: Familiarización con la interface gráfica del SNAP.
	\item Cuestionario: introducción al radar
    \vspace{1mm}
\end{itemize}
\end{minipage} \\
\midrule
\AdvanceDate[7] \mydate\today & \begin{minipage}{.80\textwidth}
\begin{itemize}
    \vspace{1mm}
	\item Video teórico: interacciones con el blanco
  \item Práctica: Interpretación visual de imágenes SAR en bandas L, C y X. Obtención de coeficientes de retrodispersión.
	\item Cuestionario: interacciones con el blanco.
    \vspace{1mm}
\end{itemize}
\end{minipage} \\

\midrule
\AdvanceDate[7] \mydate\today & \begin{minipage}{.80\textwidth}
\begin{itemize}
    \vspace{1mm}
	\item Video teórico: procesamiento y speckle.
  \item Práctica: Descarga y procesamiento de imágenes SAR del catálogo del Alaska Satellite Facility.
	\item Cuestionaario: procesamiento y speckle
    \vspace{1mm}
\end{itemize}
\end{minipage} \\


\midrule
\AdvanceDate[7] \mydate\today & \begin{minipage}{.80\textwidth}
\begin{itemize}
    \vspace{1mm}
	\item Video teórico: polarimetría.
  \item Práctica: Cálculo e interpretación de la descomposición de Pauli.
	\item Cuestionario: polarimetría.
    \vspace{1mm}
\end{itemize}
\end{minipage} \\

\midrule
\AdvanceDate[7] \mydate\today & \begin{minipage}{.80\textwidth}
\begin{itemize}
    \vspace{1mm}
	\item Video teórico: misión SAOCOM y aplicaciones.
	\item Práctica: Uso de imágenes SAR en distintas aplicaciones.
	\item Cuestionario: misión SAOCOM y aplicaciones.
    \vspace{1mm}
\end{itemize}
\end{minipage} \\

\midrule
\AdvanceDate[7] \mydate\today & \begin{minipage}{.80\textwidth}
\begin{itemize}
    \vspace{1mm}
	\item Encuesta de satisfacción del curso.
  \item Descarga de certificados de aprobación
    \vspace{1mm}
\end{itemize}
\end{minipage} \\
\midrule
\AdvanceDate[16] \mydate\today & \begin{minipage}{.80\textwidth}
\begin{itemize}
    \vspace{1mm}
	\item Fin del curso. Última fecha para completar las actividades y descargar el certificado de aprobación.
    \vspace{1mm}
\end{itemize}
\end{minipage} \\

\bottomrule
\end{longtable}
%\end{tabular}
