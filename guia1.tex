\documentclass[a4paper]{book}

\title{Curso Nivel 2a}

\begin{document}
\chapter{Guía 1}
En la primer semana del curso tenemos por objetivo
\begin{itemize}
    \item Familiarizarnos con el uso del SNAP
    \item Hacer combinaciones RGB de imágenes sentinel 2.
    \item Extraer firmas espectrales de las imágenes y compararlas.
\end{itemize}

\section{Uso del SNAP}

INTRODUCCION AL SNAP. ESA. MODULOS Y NUCLEO

PARTES DE LA INTERFACE GRAFICA.

APERTURA DE IMAGENES SENTINEL 2.

COMBINACION RGB-COLOR REAL. NAVEGAR POR LA IMAGEN

CARACTERISTICAS DE LA IMAGEN. METADATOS. VECTORES. BANDAS

BANDAS DE CALIDAD

\section{Combinaciones RGB}

COMBINACION RGB: COLOR REAL. INTERPRETACION. QUE ME DICE DE LA IMAGEN

COMBINACION RGB: INFRARROJO COLOR. INTERPRETACION. DONDE GANO. DONDE PIERDO

COMBINACION RGB: FALSO COLOR COMPUESTO. INTERPRETACION. QUE GANE DESDE EL PUNTO DE VISTA DE LA VEGETACION. QUE PERDÍ DESDE EL PUNTO DE VISTA DEL AGUA.

PARAMETROS ESTADISTICOS DE LA IMAGEN. HISTOGRAMA.

\section{Firmas espectrales}

HERRAMIENTA DE FIRMAS ESPECTRALES. QUE ESPERAMOS PARA LAS DISTINTAS ZONAS DE LA COMBINACION RGB.

PINES. PIN MANAGER. PINES EN DISTINTAS ZONAS DE LA IMAGEN.

EXTRACCION DE FIRMAS ESPECTRALES USANDO PINES.

COMPARACION DE FIRMAS ESPECTRALES. VEGETACION. AGUA. SUELO SIN COBERTURA.

FIRMAS ESPECTRALES. URBANO.

FIRMAS ESPECTRALES. VEGETACION NATURAL Y CULTIVOS.

FIRMAS ESPECTRALES. AGUA. COMPARACION ENTRE LOS RIOS.

\end{document}
