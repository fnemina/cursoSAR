\documentclass[a4paper]{article}

\title{Curso Nivel 2a}

\begin{document}
\section{Guía 3}
En la segunda semana del curso tenemos por objetivo
\begin{itemize}
    \item Comparar espectralmente imagenes corregidas y no corregidas atmosfericamente.
    \item Corregir las imagenes por el método de substracción de cuerpo obscuro.
    \item Corregir las imagenes utilizando el 6S.
\end{itemize}

\section{Comparación TOA y BOA}

COMPARACION DE FIRMAS ESPECTRALES TOA y BOA.

CALCULO DE NDVI TOA y BOA. COMPARACION.

ESTIMACION DE VARIABLES BIOFISICAS. COMPARACION.

\section{Corrección por simple dos}

CALCULO DE MINIMOS PARA LAS DISTINTAS BANDAS.

GRAFICO DEL MINIMO EN FUNCION DE LA LONGITUD DE ONDA. ESTIMACION DE LA CORRECCION PARA TODAS LAS BANDAS

CORRECCION POR SUBSTRACCION DEL MINIMO.
\section{Corrección por 6S}

OBTENCION DE PARAMETROS ATMOSFERICOS Y ESPECTRALES.

USO DEL 6S WEB.

CORRECCIÓN UTILIZANDO EL 6S-WEB

\section{Comparación entre los resultados TOA y corregidos}

FIRMAS ESPECTRALES.

CALCULO DE INDICES. DIFERENCIA PROMEDIO. HISTOGRAMA.

APLICACION DE MODELOS. RESULTADOS.

\end{document}
