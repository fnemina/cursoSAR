\documentclass[a4paper]{article}

\title{Curso Nivel 2a}

\begin{document}
\section{Guía 6}
En la sexta semana del curso tenemos por objetivo
\begin{itemize}
    \item Estudiar el efecto de incrementar el número de bandas espectrales en la clasificación.
    \item Incorporar el contexto temporal en la clasificación.
    \item Realizar transformaciones en el espacio espectral.
\end{itemize}

\section{Clasificar incrementando el numero de bandas}
CLASIFICAR POR ML UTILIZANDO 2348, INCORPORNDO LAS BANDAS DEL SWIR, INCORPORANDO LAS BANDAS DEL RED EDGE.

COMPARAR LAS MATRICES DE CONFUSION PARA LAS DISTINTAS CLASIFICACIONES.

\section{Contexto temporal}
CLASIFICAR POR ML UTILIZANDO LAS IMAGENES DE INVIERNO Y VERANO CON LAS BANDAS 2348. INCORPORAR LAS BANDAS DEL SWIR. INCORPORAR LAS BANDAS DEL RED EDGE.

\section{Contexto espacial}
INCORPORAR LA IMAGEN SENTINEL 1 SAR PARA LA ZONA DE INTERES. CALCULAR LA MATRIZ DE CO-OCURRENCIA DE GRISES. INTERPRETAR CUALITATIVAMENTE. CLASIFICAR POR KMEANS.

\section{Maxima verosimilitud}
TRANSFORMAR POR PCA LAS BANDAS DE 2348YSWIR. INTERPRETAR LAS COMPONENTES TRANSFORMADAS. CLASIFICAR UTILIZANDO DICHAS COMPONENTES(MAS IMPORTANTES).

TRANSFORMAR POR PCA LAS IMAGENES DE INVIERNO VERANO. INTERPRETAR LAS COMPONENTES. CLASIFICAR UTILIZANDO DICHAS COMPONENTES.

TRANSFORMAR POR PCA LAS IMGENES DE BANDAS Y LA IMAGEN DE TEXTURA. CLASIFICAR EL RESULTADO UTILIZANDO LAS MAS RELEVANTES.

TRANSOFRMAR POR PCA LAS IMAGENES DE INVIERNO Y VERANO Y LA IMAGEN DE TEXTURA. INTEPRETAR COMPONENTES. CLASIFICAR LOS RESULTADOS MAS RELEVANTES.

COMPARAR LAS CLASIFICACIONES OBTENIDAS POR LOS DISTINTOS METODOS. CUANTITATIVAMENTE. CALCULAR LAS AREAS Y SUS PARAMETROS ESTADISTICOS.

\end{document}
